\documentclass[11pt]{article}

%  USE PACKAGES  ---------------------- 
\usepackage[margin=0.7in,vmargin=1in]{geometry}
\usepackage{amsmath,amsthm,amsfonts}
\usepackage{amssymb}
\usepackage{fancyhdr}
\usepackage{enumerate}
\usepackage{mathtools}
\usepackage{hyperref,color}
\usepackage{enumitem,amssymb}
\newlist{todolist}{itemize}{4}
\setlist[todolist]{label=$\square$}
\usepackage{pifont}
\newcommand{\cmark}{\ding{51}}%
\newcommand{\xmark}{\ding{55}}%
\newcommand{\done}{\rlap{$\square$}{\raisebox{2pt}{\large\hspace{1pt}\cmark}}%
\hspace{-2.5pt}}
\newcommand{\HREF}[2]{\href{#1}{#2}}
\usepackage{textcomp}
\usepackage{listings}
\lstset{
basicstyle=\small\ttfamily,
% columns=flexible,
upquote=true,
breaklines=true,
showstringspaces=false
}
%  -------------------------------------------- 

%  HEADER AND FOOTER (DO NOT EDIT) ----------------------
\newcommand{\problemnumber}{0}
\pagestyle{fancy}
\fancyhead{}
%\fancyhead[L]{\textbf{Question \problemnumber}}
\newcommand{\newquestion}[1]{
\clearpage % page break and flush floats
\renewcommand{\problemnumber}{#1} % set problem number for header
\phantom{}  % Put something on the page so it shows
}
\fancyfoot[L]{IE 332}
\fancyfoot[C]{Assignment submission}
\fancyfoot[R]{Page \thepage}
\renewcommand{\footrulewidth}{0.4pt}

%  --------------------------------------------


%  COVER SHEET (FILL IN THE TABLE AS INSTRUCTED IN THE ASSIGNMENT) ----------------------
\newcommand{\addcoversheet}{
\clearpage
\thispagestyle{empty}
\vspace*{0.5in}

\begin{center}
\Huge{{\bf IE332 Project \#2}} % <-- replace with correct assignment #

Due: April 28th, 11:59pm EST % <-- replace with correct due date and time
\end{center}

\vspace{0.3in}

\noindent We have {\bf read and understood the assignment instructions}. We certify that the submitted work does not violate any academic misconduct rules, and that it is solely our own work. By listing our names below we acknowledge that any misconduct will result in appropriate consequences. 

\vspace{0.2in}

\noindent {\em ``As a Boilermaker pursuing academic excellence, I pledge to be honest and true in all that I do.
Accountable together -- we are Purdue.''}

\vspace{0.3in}

\begin{table}[h!]
  \begin{center}
    \label{tab:table1}
    \begin{tabular}{c|ccc|c|c}
      Student & Research & Implementation & Report & Overall & DIFF\\
      \hline
      Natalie & 20 & 20 & 20 & 20 & 20\\
      Jakob & 20 & 20 & 20 & 20 & 20\\
      Jack & 20 & 20 & 20 & 20 & 20\\
      Alex & 20 & 20 & 20 & 20 & 20\\
      Kento & 20 & 20 & 20 & 20 & 20\\
      \hline
      St Dev & 0 & 0 & 0 & 0 & 0\\
    \end{tabular}
  \end{center}
\end{table}

\vspace{0.2in}

\noindent Date: \today.
}
%  -----------------------------------------

%  TODO LIST (COMPLETE THE FULL CHECKLIST - USE AS EXAMPLE THE FIRST CHECKED BOXES!) ----------------------
\newcommand{\addtodo}{
\clearpage
\thispagestyle{empty}

\section*{Read Carefully. Important!}

\noindent By electronically uploading this assignment to Brightspace you acknowledge these statements and accept any repercussions if in any violation of ANY Purdue Academic Misconduct policies. You must upload your homework on time for it to be graded. No late assignments will be accepted. {\bf Only the last uploaded version of your assignment before the due date will be graded}.

\vspace{0.2in}

\noindent {\bf NOTE:} You should aim to submit no later than 30 minutes before the deadline, as there could be last minute network traffic that would cause your assignment to be late, resulting in a grade of zero. 

\vspace{0.2in}

\noindent When submitting your assignment it is assumed that every student considers the below checklist, as there are grading consequences otherwise (e.g., not submitting a cover sheet is an automatic grade of ZERO).

\begin{todolist}

    \item[\done] Your solutions were prepared using the \LaTeX template provided in Brightspace. 
    \item[\done] Your submission has a cover sheet as its first page and this checklist as its second page, according to the template provided.
	 \item All of your solutions (program code, etc.) are included in the submission as requested. % Check this checkbox and the following ones if satisfied <---
    \item You have not included any screen shots, photos, etc. (plots should be intermediately saved as .png files and then added into your .tex file). % <---
	 \item All math notation and algorithms (algorithmic environment) are created using appropriate \LaTeX code (no pictures, handwritten solutions, etc.). % <---
    \item The .pdf is submitted as an individual file and not in a {\tt .zip}.
    \item You kept the \LaTeX source code in your files until this assignment is graded, in case you are required to show proof of creating your assignment using \LaTeX.  % <---
    \item If submitting with a partner, your partner is added in the submission section in Gradescope after you upload your file. % <---
    \item You have correctly matched each question to its page \# in the .pdf submission in the Gradescope section (after you uploaded your file).
    \item Watch videos on creating pseudocode if you need a refresher or quick reference to the idea. These are good starter videos:    % <---
    
     \HREF{https://www.youtube.com/watch?v=4jLO0vXPktU}{www.youtube.com/watch?v=4jLO0vXPktU} 
    
    \HREF{https://www.youtube.com/watch?v=yGvfltxHKUU}{www.youtube.com/watch?v=yGvfltxHKUU}
\end{todolist}
}

%% LaTeX
% Für alle, die die Schönheit von Wissenschaft anderen zeigen wollen
% For anyone who wants to show the beauty of science to others

%  -----------------------------------------


\begin{document}


\addcoversheet
\addtodo

% BEGIN YOUR ASSIGNMENT HERE:
\newpage
\tableofcontents

% Algorithm Research
\newpage
\section{Sub-Algorithm Research}

%sub-algorithm one
\subsection{Fast Gradient Sign Method}

%sub-alorithm two
\subsection{L-BFGS} 
The L-BFGS method, or the Limited-Memory Broyden-Fletcher-Goldfarb-Shanno algorithm, was of interest to us in our research on performing adversarial attacks on image classifiers. Consistent with many other adversarial attack methods, L-BFGS would fool an image classifier by making small perturbations in an image. \\
First, images are converted into arrays that contains all of the pixel values; these arrays are the input of the L-BFGS algorithm. The objective function of the algorithm is designed to minimize the function maximizing the likelihood of classifier misidentification; the minimax feature of this function is designed to preserve memory and improve runtime efficiency. The algorithm then changes the pixel parameters slightly and finds how the output of the image classifier is affected before iterating this process a certain number of times (the number of iterations can be defined by the user, or it could depend on some condition, like a percentage decrease in accuracy of the image classifier's predictions.) \\
The iterations and the data obtained from them are then stored within the algorithm, and a Hessian matrix is produced from this information. The Hessian matrix is composted of second-order partial derivates that describe the behavior of every variable within an objective function. This matrix and its eigenvalues allow the model to understand how changing pixels within the training dataset affects the image classifier's output and apply this learning to images outside of the training set.

%sub-algorithm three
\subsection{Jacobian-based Saliency Map Attack}

%sub-algorithm four
\subsection{Deepfool Attack}

%sub-algorithm five:
\subsection{Zeroth-order optimization attack}
\indent A zeroth order optimization attack is a black box attack, meaning it has access to the inputs and outputs of an image classification model, but it does not have access to the architecture of the model itself.  Most black box attacks make use of a substitute model of the neural network being attacked.  This substitute model is trained using queries input into the target model and their results as training data.  The goal is to build the substitute model and then attack the substitute with a white box attack, with the resulting attack being transferrable to the initial target classification model. This type of white box attack uses the known structure of the substitute model to be able to calculate gradients using back propagation (the goal of calculating gradients is to determine which direction to move from the current pixel in order to arrive at a more important pixel).  A zeroth order optimization differs from conventional black box since it does not use a substitute model approximation of the target model.  Instead, the gradient is estimated using a first order derivative of the loss function of the target neural network.  The hessian is also estimated by taking the second order derivative of the loss function.  While this method does not give a very numerically accurate estimate of the gradient and hessian, it is good enough to be successful in the context of an attack on an image classifier.  It is worth noting that this method is required to be used in conjunction with stochastic coordinate descent (starting at a random coordinate, finding the gradient and hessian, and calculating the best movement from that coordinate to arrive at a more significant pixel) in order to create a computationally feasible solution.  Once a significant coordinate is chosen, it is updated to a new color value using an algorithm called Adam.  	\indent The amount of pixels changed, or the amount of noise in the perturbed image is conveniently easily controllable in zeroth order optimization.  This is done by transforming the space that the algorithm is searching for pixels to change into a smaller size.  This is done mainly to decrease the number of calculations required to successfully fool the image classifier, thus decreasing computing resources spent.




%sources section
\newpage
\section{Weighted Majority Classifier Research}

\newpage
\section{Implementation}

\newpage
\section{Resources}

\end{document}
